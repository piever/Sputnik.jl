\documentclass[10pt,a4paper]{report}

% mathematics packages
\usepackage{amsmath}
\usepackage{amssymb}
\usepackage{amsthm}
\usepackage{mathtools}
\usepackage{listings}
\usepackage{graphicx}
\usepackage{fullpage}
\usepackage{hyperref}
\hypersetup{
    colorlinks=true,
    linkcolor=blue,
    filecolor=magenta,
    urlcolor=cyan,
}


\begin{document}

\title{NumFocus - Julia Google Summer of Code proposal:\\
Sputink, a modern tool for data exploration based on JuliaDB and WebIO}
\author{Pietro Vertechi, mentored by Shashi Gowda}


\maketitle

\section*{Introduction}

The recent \href{https://github.com/JuliaComputing/JuliaDB.jl}{JuliaDB} package implements effective data wrangling algorithms on large datasets, potentially stored across different processors.

The package is complemented by a set of plotting recipes based on OnlineStats.jl  as well as a macro from StatPlots.jl to simplify statistical visualizations of the data: \url{http://juliadb.org/latest/api/plotting.html}.

I plan to build a web app, tentatively called Sputnik.jl, to allow users to access algorithms from JuliaDB, OnlineStats and StatPlots (also incorporating some of my prior work - \href{https://github.com/piever/GroupedErrors.jl}{GroupedErrors.jl} for analysis of population data) from a friendly user interface.

While a web app will never grant the same flexibility as coding a Julia script, I believe it has the following two advantages:

\begin{itemize}
    \item It is more inviting for users not very comfortable with coding.
    \item It simplifies completely exploratory data analysis on a dataset with a large number of columns where doing all plots by hand would be too time consuming.
\end{itemize}

I intend to integrate ideas from my previous experience building a QML-based GUI for data visualization: \href{https://github.com/piever/PlugAndPlot.jl}{PlugAndPlot.jl}. Despite having enjoyed the flexibility and features of QML while developing PlugAndPlot, this time I'd prefer to focus on a web app (based on \href{https://github.com/JuliaGizmos/WebIO.jl}{WebIO}) as it can be easily deployed:

\begin{itemize}
    \item On the plot pane in Juno (a popular Julia IDE)
    \item In a Jupyter notebook
    \item In an electron window
    \item Served in the browser
\end{itemize}

If the time allows it, I'd like to investigate whether it is feasible - for users whose data is stored on a server - to deploy such app from the server and analyze the data remotely. In that way, researchers who are willing to open-source their data could quickly set up a website where everybody can consult their data interactively (in my view, this is an excellent way to accompany a publication where only some analysis of the data are accessible).

\section*{Plan}

As a first step, I inted to port \href{https://github.com/piever/PlugAndPlot.jl}{PlugAndPlot.jl} from QML to \href{https://github.com/JuliaGizmos/WebIO.jl}{WebIO}. This will involve adding some functionality to the WebIO, InteractNext, CSSUtils stack as not all widgets and features of QML are implemented there yet. It will also be a learning opportunity for me as, despite having some experience with traditional GUI toolkits (Gtk and QML), I'm not as familiar with the recently developed WebIO stack. I will be under \href{https://github.com/shashi}{@shashi} mentoring who is one of the main developer of the WebIO stack and will help me familiarize myself with this software. \\

As a second step, I intend to optimize the analytical core of PlugAndPlot, \href{https://github.com/piever/GroupedErrors.jl}{GroupedErrors.jl}, a package which accepts any table that can iterate data, in the case where the input data is a JuliaDB table. This should be possible without sacrificing the "iterator based interface", once a set of PRs on which I'm working to collect iterators as a set of columns efficiently are merged: see:

 \url{https://github.com/JuliaComputing/IndexedTables.jl/pull/137} and

  \url{https://github.com/JuliaComputing/IndexedTables.jl/pull/135}.\\

As a third step, I intend to rethink the UI design, adding features specific to JuliaDB (such as the powerful set of online statistical analysis and visualizations - or the \href{https://github.com/JuliaComputing/TableView.jl}{TableView.jl} package, also WebIO based, to visualize the data in a spreadsheet format), incorporating feedback from the JuliaDB developers. To maintain the flexibility of working with a script, I intend to add a textbox where users can type in calls to functions from JuliaDB or \href{https://github.com/piever/JuliaDBMeta.jl}{JuliaDBMeta} to do some preprocessing on their data before visualizing it.\\

Throughout this process, I will try, as much as possible, to keep the components of the app modularized so that it would also be possible, for a user, to recombine these components to build GUIs with a different design or calling different algorithms and visualization techniques in the background.\\

Finally, if time permits, I will investigate whether it is feasible to deploy this app from a server where the data is stored, thus simplifying interactive visualizations of shared data.

\section*{About me}

Even though my background (bachelor and master) is in mathematics, I'm currently enrolled in a PhD in neuroscience. 

\end{document}
